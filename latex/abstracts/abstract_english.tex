\chapter*{Abstract}

\begin{description}

  \item[Problem statement] The huge communication barrier between
  teachers and parents who don't speak the same language is very serious
  and sensitive problem not only for the children, parents and teachers 
  but eventually also for the society. 

  As a result of difficulty in communication and integration, plenty of
  small children are a victim of isolation, which leads to some serious 
  personality and social crisis in the long run. 
  The current provincial system in place is helpful but not enough and 
  it can be improved and the help offered can be extended, by making it
  more efficient, fast and simple by minimizing the costs and resources 
  used. 
  
  Solving the issue of school-home inter-lingual and intercultural
  communication,by taking all the parties involved on board.I am targeting
  the German schools in South Tyrol,in particular the primary schools as the
  test cases.The approach used is user centered design.

  \item[Motivation] Why do we care about the problem and the results?
  If the problem isn't obviously "interesting" it might be better to put
  motivation first; but if your work is incremental progress on a problem that
  is widely recognized as important, then it is probably better to put the
  problem statement first to indicate which piece of the larger problem you are
  breaking off to work on. This section should include the importance of your
  work, the difficulty of the area, and the impact it might have if successful.

  \item[Approach] How did you go about solving or making progress on
  the problem? Did you use simulation, analytic models, prototype construction,
  or analysis of field data for an actual product? What was the extent of your
  work (did you look at one application program or a hundred programs in twenty
  different programming languages?) What important variables did you control,
  ignore, or measure?

  \item[Results] What's the answer? Specifically, most good computer
  architecture papers conclude that something is so many percent faster,
  cheaper, smaller, or otherwise better than something else. Put the result
  there, in numbers. Avoid vague, hand-waving results such as "very", "small",
  or "significant." If you must be vague, you are only given license to do so
  when you can talk about orders-of-magnitude improvement. There is a tension
  here in that you should not provide numbers that can be easily misinterpreted,
  but on the other hand you don't have room for all the caveats.

  \item[Conclusions] What are the implications of your answer? Is it
  going to change the world (unlikely), be a significant "win", be a nice hack,
  or simply serve as a road sign indicating that this path is a waste of time
  (all of the previous results are useful). Are your results general,
  potentially generalizable, or specific to a particular case?
\end{description}