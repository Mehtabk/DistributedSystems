\chapter{Introduction}
\label{ch:introduction}

You can't write a good introduction until you know what the body of the paper
says. Consider writing the introductory section(s) after you have completed the
rest of the paper, rather than before.

Be sure to include a hook at the beginning of the introduction. This is a
statement of something sufficiently interesting to motivate your reader to read
the rest of the paper, it is an important/interesting scientific problem that
your paper either solves or addresses. You should draw the reader in and make
them want to read the rest of the paper.

The next sections are important: explain \textit{why} you did this
work(motivation), textit{what} the concrete goal was (objective), and
\textit{how} you did it (approach).

\section{Motivation}
\label{sec:Motivation}

Why does it matter? Why is it important to deal with the subject you are
describing? What is the research question? Is it a good question? (has it been
answered before? is it a useful question to work on?)

\subsection{Test}

\section{Objective}
\label{sec:Objective}

This section describes the goal of the paper: why the study was undertaken, or
why the paper was written. Do not repeat the abstract and just give an overview.
You have space in chapter \ref{ch:problem} to describe the objective in
detail.

\section{Approach}
\label{sec:Approach}

What belongs in the "approach" section of a scientific paper?

\begin{tight_itemize}
  \item Information to allow the reader to assess the believability of your results.
  \item Information needed by another researcher to replicate your experiment.
  \item Description of your materials, procedure, theory.
  \item Calculations, technique, procedure, equipment, and calibration plots. 
  \item Limitations, assumptions, and range of validity. 
\end{tight_itemize}

The methods section should answer the following questions and caveats: 

\begin{enumerate}
  \item Could one accurately replicate the study (for example, all of the optional and adjustable parameters on any sensors or instruments that were used to acquire the data)? 
  \item Could another researcher accurately find and reoccupy the sampling stations or track lines?
  \item Is there enough information provided about any instruments used so that a functionally equivalent instrument could be used to repeat the experiment?
  \item If the data is in the public domain, could another researcher lay his or her hands on the identical data set?
  \item Could one replicate any laboratory analyses that were used?
  \item Could one replicate any statistical analyses?
  \item Could another researcher approximately replicate the key algorithms of any computer software?
\end{enumerate}

Citations in this section should be limited to data sources and references of
where to find more complete descriptions of procedures. Do not include
descriptions of results.

\section{Structure of the thesis}
\label{sec:StructureOfTheThesis}

State a verbal "road map" or verbal "table of contents" guiding the reader to
what lies ahead. (Something like the following paragraph.)

This thesis is divided into six sections; it starts with an introduction,
pointing out the problem, the objective and giving an overview of the applied
approach. Then it goes over to an extensive SLR revealing existing active RFID
technologies. Afterwards follows a discussion of the aim of this work. The next
step is then a description of the system's architecture illuminating the system
from different viewpoints. Furthermore, the thesis goes over to the evaluation
of the individual developed components. Finally the last section summarizes the
major activities that were conducted, their relevance and illustrates faced
technical problems and some possible future work.