\chapter{Some help on \LaTeX}

This chapter contains examples of elements you will need. You can keep this and
copy/paste from here whenever you need to use an instruction. Please read also
the instructions for each element!

\section{Text}

When you write text and you want to make a new line, make \textit{two} new lines
in the source; on the final PDF you will just see one. Within Visual Studio Code
(if you installed the re-wrap plugin), you can use \texttt{Alt+Q} to format the
text so that it fits on the screen. 

\textbf{Do not} use:

\begin{tight_enumerate}
  \item \texttt{\textbackslash\textbackslash} to make a new line.
  \item \texttt{\textbackslash newpage} to switch to a new page.
  \item \texttt{\textbackslash noindent} to avoid the indentation of a new
  paragraph.
\end{tight_enumerate}

If you use these commands, it shows that you formatted something else
incorrectly and now you want to fix a problem that is the consequence of the
first mistake. Then it is better to fix the initial mistake. Usually this is
because of a wrong positioning of figures or tables.

\section{Figures}

Figure \ref{fig:example} shows an example image. It describes various types of
data on different levels of abstraction.

\begin{figure}[ht]
	\centering
	\includegraphics[scale=0.9]{img/example.pdf}
	\caption[Caption for the index]{Caption within the text \cite{Lauesen2002}.}
	\label{fig:example}
\end{figure}

Remember: 

\begin{itemize}
  \item Figures have to be referenced in the text using \texttt{\textbackslash
  ref} and briefly explained. Do not worry if the image ends up on the following
  page, probably there is no space to have it at the place where you inserted
  it. 
  \item Images should be vector images or saved with at least 600 dpi, so that
  when printed they do not look blurry.
  \item If the caption in the text and the index are the same, you can leave the
  parameter for the caption for the index away, e.g., like
  \texttt{\textbackslash caption\{Caption within the text.\}}
\end{itemize}

\textbf{Do not} use:

\begin{tight_enumerate}
  \item \texttt{\textbackslash begin\{figure\}[H!]} to force Latex to place a
  figure where you want. Usually there is a good reason why a picture ends up
  where it is positioned. Just refer to it with \texttt{\textbackslash ref} and
  describe it. The reader is perfectly capable to find the image.
\end{tight_enumerate}

\section{Citations}
You have to define reference material in a separate file called
``bibliography.bib''. Example of citations:
\cite{Bass2012,Rubin2014,Dictionary1}. Have a look at \cite{rfc6824}.

\section{Footnotes}
Here is an example of a footnote\footnote{But do not use it too frequently :)}.
Please use footnotes to provide the web site of any technology or product, e.g.,
Microsoft Word\footnote{Microsoft Word,
\url{http://office.microsoft.com/en-us/word}}, so that everyone knows what you
are talking about.

\section{Formulas}
\LaTeX~is perfect for formulas: 
\begin{equation}
	\label{equ:formula1}
	{\frac {d}{dx}}\arctan(\sin({x}^{2}))=-2\,{\frac {\cos({x}^{2})x}{-2+\left (\cos({x}^{2})\right )^{2}}}
\end{equation}

As you see in formula \ref{equ:formula1}, you can insert very nice formulas in
your thesis too! Like figures, refer to the formula using \texttt{\textbackslash
ref} and describe what the reader is seeing.

\section{Tables}
There are many ways to make a table, the table \ref{tab:tabExample} is a bit
more complicated, but has many advantages, e.g., that you can have a table that
breaks from one page to another.

\begin{longtable}[c]{L{3cm}C{3cm}R{80pt}}
\caption{Caption of the table within the text.} 
\label{tab:tabExample} \\

\toprule
A & B & C \\
\midrule
\endfirsthead\longtableheader

\toprule
A & B & C \\
\midrule
\endhead\longtablefooter

Left aligned & Center aligned & Right aligned \\
Left aligned & Center aligned & Right aligned \\
Left aligned & Center aligned & Right aligned \\
Left aligned & Center aligned & Right aligned \\

\end{longtable}

Like for pictures and formulas, refer to the table using \texttt{\textbackslash
ref} and describe what the reader is seeing.

\textbf{Do not} use:

\begin{tight_enumerate}
  \item Vertical lines in a table
  \item Double lines in a table
\end{tight_enumerate}

\section{Code}
If you want to include code examples, you should use the \texttt{lstlisting}
environment, as in listing \ref{lst:listing1}. 

\begin{lstlisting}[caption=A listing example,label=lst:listing1]
public ArrayList getList() {	
	ArrayList l = new ArrayList();
	Connection c = null;
	try {
		c = DatabaseTools.getConnection();
		Statement p = c.createStatement();
		ResultSet r = p.executeQuery("SELECT id, \"name\", readonly FROM \"group\" ORDER BY \"name\"");
		while (r.next()) {
			HashMap h = new HashMap();
			h.put("id", r.getString(1));
			h.put("name", r.getString(2));
			h.put("readonly", new Boolean(r.getBoolean(3)));
			l.add(h);
		}
	} catch (Exception e) {
		e.printStackTrace();
	} finally {
		if (c != null) {
			try {
				c.close();
			} catch (Exception e) {
				e.printStackTrace();
			}
		}
	}
	
	return l;
}
\end{lstlisting}

Using the line numbers it is also easier to reference to them within the text.

\section{Landscape}

Sometimes, you need to show a picture or a table that requires a lot of space.
To avoid that the text in the picture or table becomes unreadable, you can
insert it in landscape mode, like the text on the next page.

\begin{landscape}
Some text in landscape mode.
\end{landscape}
