\chapter{Problem Statement}
\label{ch:problem}

% Explain the scope of your work, what will and will not be included. Is it
% obvious where introductory material ("old stuff") ends and your contribution
% ("new stuff") begins? You should organize this section by idea, and not by
% author or by publication. This section has tree main parts:

% \begin{tight_enumerate}
%   \item a concise statement of the question that your thesis tackles
%   \item justification, by direct reference to chapter \ref{ch:soa}, that your question is previously unanswered
%   \item discussion of why it is worthwhile to answer this question.
% \end{tight_enumerate}

% \section{Requirements}
% \label{sec:Requirements}

% If you are building a system, define here the concrete requirements to build it.
% Above you described which problem you want to address, here you write
% \textit{how} you designed the system to address it.

% Follow the approach of \cite{Lauesen2002}, begin with the goal requirements:

% \begin{enumerate}[label=\bfseries G\arabic*:]
%   \item The goal is a system with a flying component that is able get to a accident in a remote location and provide pictures.
% \end{enumerate}

% Then define domain requirements:

% \begin{enumerate}[label=\bfseries D\arabic*:]
%   \item The user of the system shall be able to control the flying component.
%   \item Req...
%   \item Reg222
% \end{enumerate}

% Product requirements:

% \begin{enumerate}[label=\bfseries P\arabic*:]
%   \item The user shall enter the coordinates into the ground station to which the flying component should fly.
% \end{enumerate}

% Design requirements (or constraints):

% \begin{enumerate}[label=\bfseries C\arabic*:]
%   \item The user shall enter the coordinates into the ground station to which the flying component should fly.
% \end{enumerate}

% When defining requirements, you can define functional requirements and also
% quality requirements. The quality requirements are then refined using quality
% scenarios in the next section.

% \subsection{Quality scenarios}

% Quality requirements are tricky: they are rarely detailed enough so that one can
% really understand if they are fulfilled. This is why researchers developed
% better ways to define them. Follow the approach of \cite{Bass2012}. You can use
% the scenario command as below.

% \scenario{1} %no
% {D1} %requirements
% {Consistency} %property
% {Electricity} % source of stimulus
% {Power failure} %stimulus
% {At runtime} %environment
% {Server} %artifact
% {Software has to recover all data when power comes back} %response
% {This should occur within 10 seconds} %response measure